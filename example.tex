% !Mode:: "TeX:UTF-8"
%!TEX program  = xelatex

%\documentclass[withoutpreface,bwprint]{cumcmthesis} %去掉封面与编号页

\usepackage{url}
\title{全国大学生数学建模大赛}
\tihao{A}
%% \baominghao{4321}
\schoolname{兰州大学}
\membera{马云龙}
\memberb{曾祥熙}
\memberc{俞炀}
\supervisor{无老师}
\yearinput{2019}
\monthinput{09}
\dayinput{12}

\begin{document}

\maketitle
\begin{abstract}
  摘要内容。

  \keywords{折叠桌\quad  曲线拟合\quad   非线性优化模型\quad  受力分析}
\end{abstract}

%目录
\tableofcontents

\section{问题重述}



\subsection*{问题的提出}
(1)

(2)
\section{模型的假设}

\begin{itemize}
\item 忽略实际加工误差对设计的影响;
\item 木条与圆桌面之间的交接处缝隙较小,可忽略;
\item 钢筋强度足够大,不弯曲;
\item 假设地面平整。
\end{itemize}

\section{定义与符号说明}
\begin{center}
  %此处需要画表格,问题不大
\end{center}

\section{模型的建立与求解}

\subsection*{准备工作}
\subsection*{数据处理}
\subsection*{数据采样}
\subsection*{问题一的模型}
\subsection*{问题二的模型}
\subsection*{问题三的模型}

\section{模型的检验与评价}


%参考文献
\begin{thebibliography}{9}
	\bibitem[1]{book1}文献条目一
	\bibitem[2]{book2}文献条目二
	\bibitem[3]{article1}马云龙,曾祥熙,俞炀.\emph{全国大学生数学建模论文}
\end{thebibliography}

\newpage
%附录
\begin{appendices}
	\section{C}
	\begin{lstlisting}[language=c]
	#include<stdio.h>
	int main(){
	printf("Hello world\n");
	return 0;
	}
	\end{lstlisting}
	\section{Matlab}
	\begin{lstlisting}[language = matlab]
	%% 只适用于横纵坐标是从0开始,如果不是,需要根据原理更改代码
	%% 清空变量
	clear all;
	clc;
	%% 取点之后趋势是对的,也就是点与点之间的比例是对的,但是每个点的真实值和原图对不%	上,需要按照真实的坐标处理一下
	max_axis_X=5;    %这里是真实坐标中横轴的最大值
	max_axis_Y=25;    %这里是真实坐标中纵轴的最大值
	flag = 0;    %如果图形的纵坐标是正数,flag=0,如果是负数,flag = 1;
	shift = 0;    %如果横坐标不在最下边,而是在中间,那么需要往下平移若干单位
	%% 读取图片数据
	fig=imread('D:\OneDrive\Work\MATLAB工程\3.png');    %读取图片
	imshow(fig);    %显示该图
	set(gcf,'outerposition',get(0,'screensize'));    %使该图显示最大化,便于取点
	[X0,Y0] = ginput;    %利用鼠标依次点击取出图片中数据区域边界的四个点。取点顺序:顺序%	点取图中坐标轴左下,左上,右上,右下四个点,回车结束。
	X0(2) = X0(1);    %** 对“X0,Y0”做一下处理。
	X0(4) = X0(3);
	Y0(4) = Y0(1);
	Y0(3) = Y0(2);
	[X1,Y1] = ginput;    %开始点取其中一条实线上的点,按需要的精度,点取任意多的点,回车%	结束。
	%% 因为位图的屏幕坐标是从左上角为坐标原点开始的,需要做些变换(如果图片中纵坐标不%	是从0开始,需要根据原理添加代码)
	X1 = X1 - min(X0);    %** 把图片的坐标原点移到图片中数据区域的左上角
	Y1 = Y1 - min(Y0);
	if flag == 0
	Y1= -( Y1 - (max(Y0) - min(Y0)) );    %** 把原来指向朝下的纵坐标翻转到朝上
	else
	Y1 = ( max(Y0) - min(Y0) ) - Y1;
	Y1= ( Y1 - (max(Y0) - min(Y0)) );
	end
	X1 = X1 / ( max(X0) - min(X0) ) *  max_axis_X;    %** 按照每个取得的数据点在原图片中位%	置比例换算成真实坐标
	Y1 = Y1 / ( max(Y0) - min(Y0) ) *  max_axis_Y;
	Y1 = Y1 - shift;    %往下平移若干单位
	plot(X1,Y1,'b-d');    %画图
	\end{lstlisting}
\end{appendices}


\end{document} 
